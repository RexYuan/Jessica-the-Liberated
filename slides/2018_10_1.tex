\documentclass{beamer}

\usetheme{CambridgeUS}

\usepackage[utf8]{inputenc}
\usepackage[T1]{fontenc}
\usepackage{CJKutf8}
\usepackage{datetime}
\usepackage{tikz}
\usepackage{amsmath}
\usepackage{amssymb}
\usepackage{mathtools}
\usepackage{color}
\usepackage{array}

\newcommand{\cjk}[1]{\begin{CJK}{UTF8}{bsmi}#1\end{CJK}}

\newdate{date}{01}{10}{2018}
\date{\displaydate{date}}

\begin{document}

\begin{frame}
    \frametitle{Example Table}
    \begin{center}
        \begin{tabular}{ |c|c|c|c|c| }
        \multicolumn{4}{c}{} & \multicolumn{1}{c}{\color{red}{$\downarrow$sensitive}}\\
        \hline
        name  & sex & age & passport & salary \\
        \hline
        Alice & F & 20-29 & 96273697 & 0-20k\\
        Bob   & M & 20-29 & 53331879 & 80k-90k\\
        Carol & F & 30-39 & 27429029 & 40k-50k\\
        Dave  & M & 30-39 & 53286237 & 50k-60k\\
        Eve   & F & 40-49 & 43485839 & 30k-40k\\
        \hline
        \end{tabular}
    \end{center}
\end{frame}

\begin{frame}
    \frametitle{IDentifier, as Intended}
    \begin{center}
        \begin{tabular}{ |>{\color{red}}c|c|c|>{\color{red}}c|c| }
        \hline
        name  & sex & age & passport & salary \\
        \hline
        Alice & F & 20-29 & 96273697 & 0-20k\\
        Bob   & M & 20-29 & 53331879 & 80k-90k\\
        Carol & F & 30-39 & 27429029 & 40k-50k\\
        Dave  & M & 30-39 & 53286237 & 50k-60k\\
        Eve   & F & 40-49 & 43485839 & 30k-40k\\
        \hline
        \end{tabular}
    \end{center}
\end{frame}

\begin{frame}
    \frametitle{IDentifier, as Intended}
    \begin{center}
        \begin{tabular}{ |>{\color{red}}c|c|c|>{\color{red}}c|c| }
        \hline
        name  & sex & age & passport & salary \\
        \hline
        ***** & F & 20-29 & ***** & 0-20k\\
        ***** & M & 20-29 & ***** & 80k-90k\\
        ***** & F & 30-39 & ***** & 40k-50k\\
        ***** & M & 30-39 & ***** & 50k-60k\\
        ***** & F & 40-49 & ***** & 30k-40k\\
        \hline
        \end{tabular}
    \end{center}
\end{frame}

\begin{frame}
    \frametitle{Quasi-IDentifier}
    \begin{center}
        \begin{tabular}{ |c|>{\color{red}}c|>{\color{red}}c|c|c| }
        \hline
        name  & sex & age & passport & salary \\
        \hline
        ***** & F & 20-29 & ***** & 0-20k\\
        ***** & M & 20-29 & ***** & 80k-90k\\
        ***** & F & 30-39 & ***** & 40k-50k\\
        ***** & M & 30-39 & ***** & 50k-60k\\
        ***** & F & 40-49 & ***** & 30k-40k\\
        \hline
        \end{tabular}
    \end{center}
    \alert{Scenario.} I know my friend Alice is a female in her 20s. Then I know how much she makes.
\end{frame}

\begin{frame}
    \frametitle{QID and $k$-anonymity}
    \begin{center}
        \begin{tabular}{ |c|>{\color{red}}c|>{\color{red}}c|c|>{\color{orange}}c| }
        \hline
        name  & sex & age & passport & salary \\
        \hline
        ***** & F & 20-29 & ***** & 0-20k\\
        ***** & M & 20-29 & ***** & 80k-90k\\
        ***** & F & 30-39 & ***** & 40k-50k\\
        ***** & M & 30-39 & ***** & 50k-60k\\
        ***** & F & 40-49 & ***** & 30k-40k\\
        \hline
        \end{tabular}
    \end{center}

    A \textit{quasi-identifier}(QID), a là Tore Dalenius(1986), is a set of attributes that \textbf{uniquely identifies} entries. In this example, there are two sets of QIDs: $\{sex, age\}$ and $\{salary\}$.

    A privacy property that protects against this type of re-identification is $\textit{k-anonymity}$, which, per Latanya Sweeney(2002), holds if for every entry, there are at least $k-1$ other entries with the same values for some QID.
\end{frame}

\begin{frame}
    \frametitle{$k$-anonymity Example}
    \begin{center}
        \begin{tabular}{ |c|c|c|c| }
        \hline
        name  & sex & age & salary \\
        \hline
        Frederick & M & 20-29 & 0-20k\\
        Gabriel   & M & 20-29 & 80k-90k\\
        Helen     & F & 30-39 & 40k-50k\\
        Ingrid    & F & 30-39 & 50k-60k\\
        \hline
        \end{tabular}
    \end{center}
    This table conforms to $2$-anonymity with respect to QID $\{sex, age\}$.

    \alert{Scenario.} Even if I know Frederick is a male in his 20s, I still can't be sure how much he makes.
\end{frame}

\begin{frame}
    \frametitle{\cjk{2010 年人口及住宅普查}}
    \begin{itemize}
        \item We analyzed \cjk{全臺閩地區人口狀況資料檔 (/資料檔/p99c1.txt)}.
        \item A total of 3,618,831 entries.
        \item 48 fields, whereof 39 are relavant to the analysis.
        \begin{itemize}
            \item Such as age, sex, nationality, marital status, education status...
        \end{itemize}
        \item Fields are mostly boolean or categorical.
        \item Specifications are in \cjk{/附件/99年人口及住宅普查-全臺閩地區格式}.
        \vspace{20pt}
        \item We found that of the 3,618,831 entries, there are only 1,364,624 distinct values.
        \begin{itemize}
            \item On average, for every individual, there are 2 other people that share the same values as that individual.
        \end{itemize}
    \end{itemize}
\end{frame}

\begin{frame}
    \frametitle{What We Found}
    \begin{itemize}
        \item The ``minimum'' QID, the smallest set of attributes that uniquely identifies all 1,364,624 distinct entries, is of size 35.
        \begin{itemize}
            \item \cjk{C010:性別、C020:年齡、C032:國籍、C050:婚姻狀況、C060: 與戶長關係、C071:教育狀況、C072:教育程度、C073:是否 上幼稚園(學齡前兒童)、C081:國語、C082:閩南語、C083:客 家語、C084:原住民族語、C085:其他、C091:國語、C092: 閩南語、C093:客家語、C094:原住民族語、C095:其他、C100: 五年前居住地點、C110:主要家計負責人、C121:工作狀況、C122: 行業、C123:職業、C132:工作或求學地點、C141:子女人數、C142: 子女居住地點、C151:吃飯、C152:上下床、C153:更換衣服、C154: 上廁所、C155:洗澡、C156:在室內外走動、C157:家事活動 能力、C160:是否為原住民族、C170:是否為身心障礙}.
        \end{itemize}
        \item This was found and verified in days.
    \end{itemize}
\end{frame}

\begin{frame}
    \frametitle{What We Found}
    \begin{itemize}
        \item We also found QIDs that distinguishes some percentage of the entries, that is, the QIDs that picks out some percentage of entries with distinct values.
        \item For 10\%, there are two minimum sets of QIDs of size 5. This was found and verified in days.
        \begin{itemize}
            \item \cjk{C020(年齡), C060(與戶長關係), C072(教育程度), C122(行業), C123(職業)} and
            \item \cjk{C020(年齡), C072(教育程度), C122(行業), C123(職業), C141(子女人數)}
        \end{itemize}
        \item Some small QIDs for other percentages (unverified):
    \end{itemize}
    \begin{center}
        \begin{tabular}{ c|c }
        percentage & QID size \\
        \hline
        25 & 8 \\
        50 & 10\\
        75 & 15\\
        \end{tabular}
    \end{center}
\end{frame}

\end{document}
